\section{Conclusion}
\label{sec:conclusion}

% In this paper, we looked at problem P with this context and these
% constraints. We proposed solution S. It has such good points and such not so
% good ones. 


Wollok is an educative, object-oriented programming language which is accompanied by an advanced programming environment.
Both tools are highly customized to give support to an introductory OOP course.
Our approach consists of the combination of these three cornerstones.

First, an \emph{incremental\np{Revisar la idea de Javi} learning path} choosing exactly which are the concepts we want to teach and the order we want to teach them.
The learning path starts with a \emph{simplified programming model} (SPM), \ie one which uses less concepts than a full-fledged OOP language.
The SPM allows the student to build simple programs without requiring more advanced concepts.
The path should attach the SPM with a good set of programming exercises, specifically oriented to be easy to build in the selected SPM.
Once the student has mastered the concepts on the SPM, we can go a step further and introduce the next set of concepts.
\np{Acá se podría hablar de constructivismo.}

Next, defining our own programming language, allows us to give full support to the selected learning path, 
avoiding the need of explaining complex concepts too soon in the course or forcing the student to write boilerplate code which he cannot yet understand.

Finally, a good programming environment, helps detecting errors, provides guidance and most significantly allows the student to \emph{explore}.
We have found that often students are afraid to search for solutions not seen in the class or test their own ideas, 
which leads them to restricting themselves into a smaller set of concepts and tools they feel more secure about.
A controlled environment empowers students to look around and explore new possibilities.

\medskip

% Now we could do this or that.
\label{sec:furtherWork}
One major objective in our future work is the integration of more \emph{automatic user interaction} tools into the Wollok environment.
Our objective is to enable the students to have visual and interactive programs without requiring them to learn the subleties of GUI building, 
extending the ideas in Gobstones \cite{lopez_nombre_2012} to object-oriented domains.
Some advances in this area can be seen in our previous work named Hoope \cite{estefania_miguel_hoope_2013}.

The second major objective is to continue improving the detection of programming errors.
A cornerstone to achieve this goal is the type inferer, which is our current focus.
The other half of our future work in this area is a powerfull \emph{effect system} \cite{nielson_type_1999}.

Another characteristic of programming in the real world is the need to work in teams. 
The success of object-oriented languages is partly due to their advantages in group projects. 
It is necessary teach our students about the techniques needed for teamwork, right from the beginning. 
To do this, it is essential that the environment has some form of support for group work \cite{kolling_problem_1999}.
Therefore, we plan to create simplified tools to integrate wollok te \emph{version control systems}.

Also we are working in adding more automatic refactor tools, and a better type inference implementation. Even working on adding an effect system to detect correct usage of the language and the code conventions.

One of the important development steps to be done is the implementation of a web version or a lighter version, using less hardware requirements, with the aim to run the solution in small netbooks like the ones in the program Conectar Igualdad \footnote{http://www.conectarigualdad.gob.ar/}

Finally, in the educational use of the tool, we will be testing it in different educational environments to get feedback about the learning experience; generating learning material (\eg examples, exercises, guides). As the focus of the tool is to provide a new way of teaching programming skills. For this objective, we will be working in collaboration with Universities, Teachers and non profit organizations.
