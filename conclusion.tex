\section{Conclusion}
\label{sec:conclusion}

% In this paper, we looked at problem P with this context and these
% constraints. We proposed solution S. It has such good points and such not so
% good ones. 


Wollok is an educative, object-oriented programming language which is accompanied by an advanced programming environment.
Both tools are highly customized to give support to an introductory OOP course.
Our approach consists of the combination of these three cornerstones.

First, an \emph{incremental\np{Revisar la idea de Javi} learning path} choosing exactly which are the concepts we want to teach and the order we want to teach them.
The learning path starts with a \emph{simplified programming model} (SPM), \ie one which uses less concepts than a full-fledged OOP language.
The SPM allows the student to build simple programs without requiring more advanced concepts.
The path should attach the SPM with a good set of programming exercises, specifically oriented to be easy to build in the selected SPM.
Once the student has mastered the concepts on the SPM, we can go a step further and introduce the next set of concepts.
\np{Acá se podría hablar de constructivismo.}

Next, defining our own programming language, allows us to give full support to the selected learning path, 
avoiding the need of explaining complex concepts too soon in the course or forcing the student to write boilerplate code which he cannot yet understand.

Finally, a good programming environment, helps detecting errors, provides guidance and most significantly allows the student to \emph{explore}.
We have found that often students are afraid to search for solutions not seen in the class or test their own ideas, 
which leads them to restricting themselves into a smaller set of concepts and tools they feel more secure about.
A controlled environment empowers students to look around and explore new possibilities.

\medskip

% Now we could do this or that.

\label{sec:furtherWork}
Y a futuro le agregaríamos:
- Integración con la UI, que es otra línea de trabjo que estuvimos trabajando, tirar link a Hoope, en algún momento tenemos que integrar las dos ideas. También se puede decir que tomamos como base Gobstones.
- Integración fácil con SCM

Another characteristic of programming in the real world is the need to work in
teams. The success of object-oriented languages is partly due to their advantages in
group projects. Ideally, we also want to teach our students about the techniques
needed for teamwork. To do this, it is essential that the environment has some form
of support for group work. \cite{kolling_problem_1999}

- Más refactors y mejora del sistema de inferencia.
- Una versión web / versión liviana con el objetivo de poder ejecutarse en las netbooks que tienen los chicos de secundaria.

Probar la nueva herramienta en entornos educativos.

Eso pensando el lenguaje/herramienta, si pienso a nivel docencia se me ocurre que lo que tenemos que hacer es integrarnos con otras entidades, como Sadosky u otras universidades.
Haciendo foco en que el punto no es la herramienta sino que tenemos que repensar cómo enseñamos a programar.

la idea de hacer un effect system
power que detecte efecto de lado, y asi poner checkeos para resolver el problema de si un método es una 'orden' o una 'pregunta', 

