\section{Conclusion}
\label{sec:conclusion}

% In this paper, we looked at problem P with this context and these
% constraints. We proposed solution S. It has such good points and such not so
% good ones. 

Más aún, que el terreno sea acotado fomenta que ellos lo puedan explorar. 
Habrás notado que cuesta que los estudiantes se salgan del caminito que uno mostró en clase, que busquen la documentación de collection por sí mismos. Bueno eso podría tener muchas razones pero yo creo que una que podemos atacar es que cuando hacen eso normalmente esa documentación los marea y huyen.

Ergo para mí es clave 
- elegir exactamente los conceptos que se quieren mostrar (y tal vez otros que no se muestran pero que se dejan al alcance de la mano para que loos estudiantes con más inquietudes los encuentren solos).
- tener un lenguaje propio que sólo tenga lo que nosotros queremos mostrar (y un entorno que permita explorar lo que hay).
- tener un IDE que favorezca la detección de errores (y favorezca la exploración).

% Now we could do this or that.

Y a futuro le agregaríamos:
- Integración con la UI, que es otra línea de trabjo que estuvimos trabajando, tirar link a Hoope, en algún momento tenemos que integrar las dos ideas. También se puede decir que tomamos como base Gobstones.
- Integración fácil con SCM

Another characteristic of programming in the real world is the need to work in
teams. The success of object-oriented languages is partly due to their advantages in
group projects. Ideally, we also want to teach our students about the techniques
needed for teamwork. To do this, it is essential that the environment has some form
of support for group work. \cite{kolling_problem_1999}

- Más refactors y mejora del sistema de inferencia.
- Una versión web / versión liviana con el objetivo de poder ejecutarse en las netbooks que tienen los chicos de secundaria.

Probar la nueva herramienta en entornos educativos.

Eso pensando el lenguaje/herramienta, si pienso a nivel docencia se me ocurre que lo que tenemos que hacer es integrarnos con otras entidades, como Sadosky u otras universidades.
Haciendo foco en que el punto no es la herramienta sino que tenemos que repensar cómo enseñamos a programar.
