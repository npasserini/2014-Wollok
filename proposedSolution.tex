\section{Wollok: The Language}
\label{sec:wollokLanguage}

% Free form, variable number of sections, technical details.
% But in general do not mix solution and discussions/possible variation let that for discussion

Wollok is a brand new language, built to specifically give support to our pedagogical approach.
Many ideas have been inherited from our previous projects, such as Ozono or Loop. 
The most important of these inherited characteristics is the ability to create objects and treat them polymorphically without the need of type annotations, classes or inheritance.
Also, as its predecessors and unlike other pedagogical tools, Wollok is a \emph{general purpose} language, \ie it is not tied to any specific domain.

\medskip
One of the main objectives of building a new language is to provide a smoother transition from the first phase of the course,
in which students use a simplified programming modelm and the second phase, in which they use a full-fledged OOP language.
With our our previous programming tools students were required to discard, in the middle of the course, the programming model and enviromnent they already knew and were used to.
This transition has sometimes been traumatic, because the process to define an object has to be re-learnt and the tools they had been using up to that point are no longer available.
With our new approach, the tools they first learnt are going to continue being available through all the course, together with the new ones that are incorporated in later lectures.
This is also consistent with some modern industrial OO languages that allow to define both classes or standalone objects, such as Scala \cite{Oder04a}.

\np{Acá tal vez podríamos poner un ejemplo que muestre eso... aunque me queda un poco desordenado poner esto antes del ejemplo más básico, escucho propuestas.}

\medskip
Also we reduced to a minimum the syntax and the most basic constructs of the language.
While this objective was already present in our previous work, the implementation strategy of Wollok \cf \ref{sec:implementation} allows us to go much further in accomplishing this goal.
The example in figure \ref{fig:helloWorld/wollok} shows how would look the classical hello world in Wollok.
To build this first program students are not required to know about typing, scoping or packaging.
The only required construct is the \lstinline[language=Wollok]{program} and the only command is a message send.
Both the receiver and the parameters are built-in objects which will be handled in the same way user-defined objects.
The concepts required to understand this program are no more than program, object, message and argument passing.

\begin{figure}[h]
 \centering
 \begin{lstlisting}[language=Wollok]
	program {
		console.println("Hello World!")
	}
 \end{lstlisting}
 
 \caption{\small Sample initial Wollok program.}
 \label{fig:helloWorld/wollok}
\end{figure}

\np{No estoy seguro de cómo meter lo que sigue, requiere de una explicación más larga.}
\begin{itemize}
\item \textbf{Profundizar y pulir el highlighting the conceptos primarios y la
estratificacion de conceptos}.
	(ej: literales de objetos, literales de colecciones. Objetos no como un
	elemento de la IDE -Ozono: nueva referencia global-, sino como un elemnto del
	lenguaje. Evita referencias globales.)
\item \textbf{Introducción de nuevos elementos concretos que explicitan
conceptos ya existentes} (ej: 1- var/val, 2- la idea de hacer un effect system
power que detecte efecto de lado, y asi poner checkeos para resolver el problema de si un método es una 'orden' o una 'pregunta', 3- program/libreria/test, 4-override ).
\end{itemize}

- bueno, muchas mejoras a nivel lenguaje, como literales para coleciones, imports


\section{A Customized Programming Environment}
\label{sec:environment}

Beginner programmers are likely to require more guidance and make more mistakes than experienced programmers.
Therefore, we think that is much to gain from a good programming environment which structures the programming experienced and helps the students to identify common mistakes.

% \subsubsection{Visualización y Navegación}
% \begin{itemize}
%   \item syntax highlight
%   \item outline
%   \item hovering
%   \item vista de problems 
%   \item navigate: goto (F3, click), flechita para ir al método que sobrescribe.
%   \item find references 
% \end{itemize}
% 
% \subsubsection{Asistencia}
% \begin{itemize}
%   \item content assist
%   \item quick fixes
%   \item code templates (nuevo)
% \end{itemize}
The Wollok programming environment includes a lot of features that provide guidance to the student.
\emph{Content assist} shows the students what are his possibilities at any moment and feeds automatically into the code the most usual constructs, 
allowing the student to concentrate less on syntax and more in the modelling of the exercise problem.
\emph{Advanced code navigation} and \emph{smart reference searches} allow the programmer to better understand the dependencies in his program.
\np{¿Se les ocurre cómo mejorar eso?}
Moreover, \emph{automatic class diagrams} provide a high level view of the program and also helps understanding.

% Detect mistakes
Also, the programming environment has many tools intended to help detecting mistakes, even while the student is writing code.
\emph{Syntax highlighting} helps identify the most simple mistakes by providing immediate feedback when something is not right. 
Moreover, the enviromnent provides \emph{real-time highlights} for syntactic mistakes.
Finally, the \emph{type inferer} allows to detect more subtle mistakes.
All these tools allows the student to gain more control of his code, keeping him away from feeling lost, 
which is otherwise a common situation for a student walking his first steps into programming.

% Este no sé cómo ponerlo, es muy crítica al smalltalk.
% 8-reducir errores frustrantes: se cancela la edicion por tener 1 solo editor de metodo por ves (poder visualizar más que un sólo método simul), evitar errores de imagenes)

\medskip
% Type inferer
The type inferer is one of the most distintive characteristics of the Wollok programming environment.
We think that type inference is key to a simple programming environment.
On one side, it allows to detect lots of common mistakes \emph{before running the program}:
if an object does understand a message, if a wrong argument is passed, if incompatible types are mixed or even miss-spellings.
In enviroments without this capability it takes more time to detect errors.
Moreover, it is not uncommon that a type mistake produces a runtime error in a place different from where the mistake was done, producing confussion.

Still, providing a type inferer for a language such as Wollok has its subleties, which deserves an independent study \cite{type inferer}.
\np{no sé qué más decir de esto}

\subsubsection{Checkeos y validaciones}
 
Todos los checkeos y problemas
generados se muestran agrupados en una vista dedicada a tal fín (Problems).

% En lo que sigue fui comentando las cosas que ya están dichas pero no quiero borrar esta enumeración porque está muy buena.
\begin{itemize}
%   \item \textbf{De sintaxis}: dados por el parser y lexer automáticamente.
  \item \textbf{De estilo}: para promover uniformidad y consistencia de código.
  Ejemplos:
  		\begin{itemize}
  			\item \textit{Nombres}: variables camelcase comenzando en minúscula,
  			nombres de clases camelcase iniciando mayúscula, packages en minúsculas, etc.
  			\item \textit{Orden y agrupamiento}: dentro de un objeto o clase, primero
  			se definen sus referencias internas, luego constructores y finalmente los métodos.
  			\item \textit{Separación de programas}: las clases sólo se pueden definir
  			en archivos de tipo \textit{librería}, no dentro de un \textit{program}.
  			\item \textit{Evitar referencias duplicadas}: no se puede definir una
  			referencia con nombre ya utilizado en alguna otra referencia del contexto (local, método,
  			clase/objeto, etc.). Ni tampoco si ya está definida en la superclase.
		\end{itemize}
  \item \textbf{De resolución de referencias}: para evitar referencias a
  variables inexistentes y, en la medida de lo posible (por ser de tipado
  implícito) de envío de mensajes. Ejemplos:
  		\begin{itemize}
		  \item \textit{Referencias inexistentes}: a variables locales, parámetros, o
		  internas (clase/objeto).
		  \item \textit{Constructores inexistentes}: evaluando existencia de la
		  clase, y compatibilidad en el número de paråmetros.
		  \item \textit{Envío de mensajes (a this)}: al ser a this se pueden realizar
		  checkeos por la existencia del método y compatibilidad de parámetros, incluso sin
		  involucrar al sistema de tipos.
		\end{itemize}
  \item \textbf{De uso de referencias}: para la detección de código
  	erroneo o bien desactualizado. Por ejemplo: warnings por referencias nunca
 	utilizadas, nunca asignadas, o utilización de variables en lugar de valores.
  \item \textbf{De estructura}: evitan por ejemplo inconsistencias en las
  estructuras creadas por el alumno. Por ejemplo, se checkea
  que un método marcado como \textit{override} efectivamente esté
	sobrescribiendo.
  \item \textbf{De tipos}: verifican compatibilidad de referencias en base a sus
  tipos. Por ejemplo ante envío de mensajes, o asignaciones de variables. Basado
  en el sistema de tipos.
\end{itemize}


