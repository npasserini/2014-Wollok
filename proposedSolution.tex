\section{Wollok: the language}
\label{sec:wollokLanguage}

% Free form, variable number of sections, technical details.
% But in general do not mix solution and discussions/possible variation let that for discussion

Wollok is a brand new language, built to specifically give support to our pedagogical approach.
Many ideas have been inherited from our previous projects, such as Ozono or Loop. 
The most important of these inherited characteristics is the ability to create objects and treat them polymorphically without the need of type annotations, classes or inheritance.
Also, as its predecessors and unlike other pedagogical tools, Wollok is a \emph{general purpose} language, \ie it is not tied to any specific domain.

\medskip
One of the main objectives of building a new language is to provide a smoother transition from the first phase of the course,
in which students use a simplified programming modelm and the second phase, in which they use a full-fledged OOP language.
With our our previous programming tools students were required to discard, in the middle of the course, the programming model and enviromnent they already knew and were used to.
This transition has sometimes been traumatic, because the process to define an object has to be re-learnt and the tools they had been using up to that point are no longer available.
With our new approach, the tools they first learnt are going to continue being available through all the course, together with the new ones that are incorporated in later lectures.
This is also consistent with some modern industrial OO languages that allow to define both classes or standalone objects, such as Scala \cite{Oder04a}.

\np{Hasta acá llegué}

\begin{itemize}
\item \textbf{Profundizar y pulir el highlighting the conceptos primarios y la
estratificacion de conceptos}.
	(ej: literales de objetos, literales de colecciones. Objetos no como un
	elemento de la IDE -Ozono: nueva referencia global-, sino como un elemnto del
	lenguaje. Evita referencias globales.)
\item \textbf{Introducción de nuevos elementos concretos que explicitan
conceptos ya existentes} (ej: 1- var/val, 2- la idea de hacer un effect system
power que detecte efecto de lado, y asi poner checkeos para resolver el problema de si un método es una 'orden' o una 'pregunta', 3- program/libreria/test, 4-override ).
\item \textbf{Unificar las fases del aprendizaje} (ej: objetos+clases: un solo
lenguaje, misma herramientas, poder reutilizar y hacer convivir)
\end{itemize}

\begin{itemize}
\item \textbf{Proveer un entorno inteligente que}: por un lado, estructure en
forma más estricta/explícita la experiencia; y que, por el otro lado, permita una gran asistencia al estudiante/desarrollador (esto tiene muchos elementos: 1- desde content assist, 2-syntax coloring, 3- resaltado de errores (sintaxis y tipado) 4-navegación de código, 5-busqueda de referencias, 6-diagramas automáticos de clases, 7-hasta llegar un sistema de tipos que permita la detección temprana de errores, 8-reducir errores frustrantes: se cancela la edicion por tener 1 solo editor de metodo por ves (poder visualizar más que un sólo método simul), evitar errores de imagenes)
\item \textbf{Acercar la experiencia de aprendizaje a las prácticas
industriales}: (acá el palo de que la imagen sólo existe en smalltalk, y en la
industria nadie la usa. Atrás de eso, la idea de archivos, y poder compartir con SVC. Por último la idea de actualizarse a un lenguaje con influencia de lenguajes modernos como xtend, scala, ruby, etc.)
\end{itemize}
- bueno, muchas mejoras a nivel lenguaje, como literales para coleciones, imports


\subsection{Entorno Inteligente de Trabajo}
\label{subsec:proposedSolution-entornoInteligente}

\subsubsection{Visualización y Navegación}

// todo: 
\begin{itemize}
  \item syntax highlight
  \item outline
  \item hovering
  \item vista de problems 
  \item navigate: goto (F3, click), flechita para ir al método que sobrescribe.
  \item find references 
\end{itemize}

\subsubsection{Asistencia}

//todo: 
\begin{itemize}
  \item content assist
  \item quick fixes
  \item code templates (nuevo)
\end{itemize}

\subsubsection{Checkeos y validaciones}

Wollok provee numerosos checkeos y validaciones estáticas, a fin de que el
alumno pueda encontrar los problemas en su código de manera temprana.
Incluso a medida que va escribiendo su código.
Esto contribuye a que el alumno mantenga control completo de su
código, y evita la sensación de estar perdido.
 
Todos los checkeos y problemas
generados se muestran agrupados en una vista dedicada a tal fín (Problems).

\begin{itemize}
  \item \textbf{De sintaxis}: dados por el parser y lexer automáticamente.
  \item \textbf{De estilo}: para promover uniformidad y consistencia de código.
  Ejemplos:
  		\begin{itemize}
  			\item \textit{Nombres}: variables camelcase comenzando en minúscula,
  			nombres de clases camelcase iniciando mayúscula, packages en minúsculas, etc.
  			\item \textit{Orden y agrupamiento}: dentro de un objeto o clase, primero
  			se definen sus referencias internas, luego constructores y finalmente los métodos.
  			\item \textit{Separación de programas}: las clases sólo se pueden definir
  			en archivos de tipo \textit{librería}, no dentro de un \textit{program}.
  			\item \textit{Evitar referencias duplicadas}: no se puede definir una
  			referencia con nombre ya utilizado en alguna otra referencia del contexto (local, método,
  			clase/objeto, etc.). Ni tampoco si ya está definida en la superclase.
		\end{itemize}
  \item \textbf{De resolución de referencias}: para evitar referencias a
  variables inexistentes y, en la medida de lo posible (por ser de tipado
  implícito) de envío de mensajes. Ejemplos:
  		\begin{itemize}
		  \item \textit{Referencias inexistentes}: a variables locales, parámetros, o
		  internas (clase/objeto).
		  \item \textit{Constructores inexistentes}: evaluando existencia de la
		  clase, y compatibilidad en el número de paråmetros.
		  \item \textit{Envío de mensajes (a this)}: al ser a this se pueden realizar
		  checkeos por la existencia del método y compatibilidad de parámetros, incluso sin
		  involucrar al sistema de tipos.
		\end{itemize}
  \item \textbf{De uso de referencias}: para la detección de código
  	erroneo o bien desactualizado. Por ejemplo: warnings por referencias nunca
 	utilizadas, nunca asignadas, o utilización de variables en lugar de valores.
  \item \textbf{De estructura}: evitan por ejemplo inconsistencias en las
  estructuras creadas por el alumno. Por ejemplo, se checkea
  que un método marcado como \textit{override} efectivamente esté
  sobrescribiendo.
  \item \textbf{De tipos}: verifican compatibilidad de referencias en base a sus
  tipos. Por ejemplo ante envío de mensajes, o asignaciones de variables. Basado
  en el sistema de tipos.
\end{itemize}


