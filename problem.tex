\section{Problem Description}
\label{sec:problem}

% Context, exposed with the \textbf{most precise terms possible} (don't open
% unwanted doors for the reader)

% Probably set the vocabulary before to cut any misinterpretation

% Constraints that influenced the solution (because the solution is not
% universal) \emph{e.g.} our requirements for a solution, possibly not all
% satisfied. They should be sound and believable. Analysis of the criteria.
% Imagine that you are another guy having this problem do the constraint
% matches yours so that you could apply the solution

% Problem
These difficulties become even bigger for students which have learned the imperative model first,
and object-oriented material is delayed to intermediate courses. \cite{cooper_teaching_2003}.

One of the causes of the difficulties to learn OOP is that courses often focus on syntax and the particular characteristics of a
programming language, leading students to concentrate on these relatively unimportant details rather than the underlying algorithmic skills. 
By focusing on details, many students fail to comprehend the essential model that transcends particular programming languages\cite{the_joint_task_force_on_computing_curricula_computing_2001}. 
Moreover, most of the object-oriented languages used in introductory courses require the student to grasp several concepts before being able to write his first program.
For all these reasons, courses tend to spend too much time concentrated on the mechanistic details of programming constructs, 
leaving too little time to become fluent on the distinctive characters of OOP, 
such as identifying objects and their knowledge \emph{relationships}, assigning \emph{responsibilities} 
and taking advantage of \emph{encapsulation} and \emph{polymorphism} to make programs more robust and extensible.

The failure of students to understand the essential object-oriented concepts shows both in academy as in industry.
In academic environments we find very low completion rates in introductory OO courses.
Moreover, students in their very beginning of an informatic carreer which fail their introductory programming courses, are often likely to give up their studies\np{cite needed}.
In industrial development, we find that these hindrances reduce the opportunity of students to apply
the concepts of the paradigm effectively in their further
professional practice, resulting in several IT-projects not taking
advantage of the possibilities offered by the potential of good
object-oriented practices. \cite{lombardi_instances_2007}

This means students must dive
right into classes and objects, their encapsulation (public
and private data, etc.) and methods (the constructors,
accessors, modifiers, helpers, etc.). All this is in addition to
mastering the usual concepts of types, variables, values, and
references, as well as with the often-frustrating details of
syntax. Now, add event-driven concepts to support
interactivity with GUIs! As argued by [11], learning to
program objects-first requires students grasp "many
different concepts, ideas, and skills…almost concurrently.
Each of these skills presents a different mental challenge." \cite{cooper_teaching_2003}

% Factual solution tracks, to position...

Already in 1999, Kolling \etal had established the importance of enviroments in introductory courses\cite{kolling_problem_1999}. 

better environments have become necessary. Earlier introductory courses
focused on the development of algorithms in procedural or functional languages. To
do this, an editor and a compiler was all that was needed for the practical part of the
work. Modern courses now use object-oriented languages and subject material
taught includes testing, debugging and code reuse. This creates the need to deal with
multiple source files and multiple program development tools from the very start. To
give a beginning student a chance to cope with this increased complexity, better
environment support is needed\cite{kolling_problem_1999}. 

Ozono hereda muchos de los problemas de los entornos Smalltalk

Smalltalk, however, lacks other important facilities: no visualisation tools for class
relations are available. The main problem with this lies in the Smalltalk language
itself: since it is not statically typed, it is not possible to extract usage relations from
its source code. No indication exists before runtime as to the call relationships
between classes. Inheritance relationships as shown in the browser do not present
the relationships of one application but rather the whole Smalltalk environment and
so the browser is not used as an application modelling tool. Smalltalk blurs the
distinction between the environment and the application under development.
Reports about the use of Smalltalk systems for teaching also point to another
problem: its size. While the language itself (in terms of the number of constructs) is
small, the class library and tools are large and often confusing. Several authors
reported difficulties with the students ability to cope with the environment [7, 8],
especially that experimentation and self directed learning was not working well
because students were overwhelmed by the system. They also found that the
functionality of the browser should be limited, since its power and flexibility caused
more problems than it solved. \cite{kolling_problem_1999}.

% Our solution in a nutshell.

