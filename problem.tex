\section{Problem Description}
\label{sec:problem}

% Context, exposed with the \textbf{most precise terms possible} (don't open
% unwanted doors for the reader)

% Los cursos se enfocan en sintaxis y usan lenguajes inadecuados.
On cause behind the difficulties in learning OOP is that courses often focus on syntax and the particular characteristics of a
programming language, leading students to concentrate on these relatively unimportant details rather than the underlying object-oriented modelling skills. 
Also, the use of industrial languages to introduce OOP requires the student to understand several concepts before being able to run his first program. \cite{kolling_problem_1999}

% Ejemplo con Java.
By focusing on those details, many students fail to comprehend the essential model that transcends particular programming languages \cite{the_joint_task_force_on_computing_curricula_computing_2001}. 
Figure \ref{fig:helloWorld} shows a typical first program, written in Java \cite{arnold_java_1996}.
To get this program running, the student has to walk through a minefield of packages, classes, scoping, types, arrays and class methods among other complex concepts.
However, he is still not able to do more basic OO programming, such as sending a message to an object.

\begin{figure}[ht]
 \centering
 \begin{lstlisting}[language=Java]
	package examples;
	
	public class HelloWorld {
		public static void main(String[] args) {
			System.out.println("Hello World");
		}
	}
 \end{lstlisting}
 
 \caption{\small Sample initial Java program which diverts student attention from the most important concepts.}
 \label{fig:helloWorld}
\end{figure}

% Por eso los pibes no aprenden
For all these reasons, courses tend to spend too much time concentrated on the mechanistic details of programming constructs, 
leaving too little time to become fluent on the distinctive characters of OOP, 
such as identifying objects and their knowledge \emph{relationships}, assigning \emph{responsibilities} 
and taking advantage of \emph{encapsulation} and \emph{polymorphism} to make programs more robust and extensible.

% Además necesitamos environments
To make things worse, frequently the students do not have proper tools that could help them overcoming all the obstacles.
Already in 1999, Kolling \etal had established the importance of environments in introductory courses \cite{kolling_problem_1999}. 
He stated that earlier introductory courses focused on the development of algorithms in procedural or functional languages. 
To do this, an editor and a compiler was all that was needed for the practical part of the work. 
Modern courses now use object-oriented languages and subject material taught includes testing, debugging and code reuse. 
This creates the need to deal with multiple source files and multiple program development tools from the very start. 
To give a beginning student a chance to cope with this increased complexity, better environment support is needed. 

\medskip

% Probably set the vocabulary before to cut any misinterpretation
% Problem

% Consecuencias en la acadamia y en la industria
The failure of students to understand the essential object-oriented concepts shows both in academy as in industry.
In academic environments we find very low completion rates in introductory OO courses.
Moreover, students in their very beginning of an informatics career which fail their introductory programming courses, are often likely to give up their studies\cite{milne_2002}.
In industrial development, we find that these hindrances reduce the opportunity of students to apply
the concepts of the paradigm effectively in their further
professional practice, resulting in several IT-projects not taking
advantage of the possibilities offered by the potential of good
object-oriented practices. \cite{lombardi_instances_2007}

\medskip 

% Factual solution tracks, to position...
\np{Contar algo de otras propuestas anteriores a la nuestra}

Our previous work in this area is based on the proposal to change the order of subjects in OOP introductory coures \cite{lombardi_instances_2007}.
To enable this we provide a simplified OO language with an ad-hoc educational programming environment 
named Ozono\footnote{The name has been changing along these years. Other names for our programming enviroment have been ObjectBrowser, Loop and Hoope}, 
in which students can create objects and define their behavior directly, without the need for classes and inheritance \cite{griggio_programming_2011}.
Ozono is based on Smalltalk\footnote{The latest version currently in use in universities is based on a Smalltalk dialect named Pharo \cite{black_pharo_2010}.}.
and thus it is a \textit{dynamic language}.

We take special advantage of the dynamic characteristic of the language because it allows very simple uses of polymorphism, 
\ie any two objects that understand a common set of messages can be treated polymorphically without worrying about inheritance or \emph{interface} implementation.
This enables the students to exploit polymorphism already in their programs after only two or three lectures.
\np{Qué otras ventajas queremos contar?}

\medskip
This approach produced impressive results and therefore has been adopted in other courses and universities.
The simplified programming model, allows the students to spend more time working with the essential concepts of the OO paradigm.
This, in turn allows for more practice and more complex exercises, even incorporating more advanced OO techniques such as \emph{design patterns} \cite{Gamm03a},
which in traditional approaches are normally postponed to a second OOP course.

% Esto podría extender la descripción del problema
%  Ozono hereda muchos de los problemas de los entornos Smalltalk

% Smalltalk, however, lacks other important facilities: no visualisation tools for class
% relations are available. The main problem with this lies in the Smalltalk language 
% itself: since it is not statically typed, it is not possible to extract usage relations fromo
% its source code. No indication exists before runtime as to the call relationships
% between classes. Inheritance relationships as shown in the browser do not present
% the relationships of one application but rather the whole Smalltalk environment and
% so the browser is not used as an application modelling tool. .
% Reports about the use of Smalltalk systems for teaching also point to another
% problem: its size. While the language itself (in terms of the number of constructs) is
% small, the class library and tools are large and often confusing. Several authors
% reported difficulties with the students ability to cope with the environment [7, 8],
% especially that experimentation and self directed learning was not working well
% because students were overwhelmed by the system. They also found that the
% functionality of the browser should be limited, since its power and flexibility caused
% more problems than it solved. \cite{kolling_problem_1999}.


% Our solution in a nutshell.
% (c) las ideas nuevas... y wollok como herramienta para dar soporte a esas ideas.

% Our solution: approach
Still, the experience of eight years in four universities\footnote{Ozono is currently used in Universidad Tecnológica Nacional, F. R. Buenos Aires and F.R. Delta, 
Universidad Nacional de Quilmes, Universidad Nacional de San Martín and Universidad Nacional del Oeste.}
and thousands of students has provided us with new insights of the learning process, 
which is what lead us to introduce several adjustements in our approach.

In the first place, starting with a \emph{classless} language mandates a change of language and enviroment in the middle of the course,
which is confusing for some students. 
To solve this, we propose to integrate \emph{class-based} and \emph{self-defined} objects in the same language.
\np{Ver bien qué nombres les ponemos y definir los conceptos antes.}

Second, we think that the absence of static typing information (which Ozono inherits from Smalltalk) prevents our programming environment to aid the students in finding some typical beginner mistakes.
Still, we do not want to force the students to add type annotations to their code, because that would distract them from the concepts we want to focus on.
We propose to settle this apparent controversy by enhancing our programming environment with a type inferer.
By doing so, the student is not required to care about type annotations, and at the same type we can detect some programming mistakes and aid the student in solving them.

Last, we have detected that sometimes the students which seem to understand the main concepts and can apply them in interesting ways to create medium to complex program, 
then have a hard time translating this knowledge to their professional activity.
We think that a good mitigation plan for this problem starts with bringing the activities in the course as close as possible to the professional practice.
For that matter, we propose to incorporate industrial best practices such as code repositories and unit tests, adapted to the possibilities of students with little or no programming experience.

\medskip
% Our solution: environment
The renewed approach is supported with a new programming language, named Wollok, and a programming environment which aids students to write, test and run programs.
Wollok is designed to give support to our pedagogical approach; 
it allows to define both classes and standalone objects,
incorporates a basic type inferer
and provides a simple syntax to define unit tests.

A big difference with Ozono and other educational OO languages is that Wollok is a \emph{file-based} language.
While we recognize the value of the \emph{image-based} approaches, we also are aware that most industrial programming enviroments are file-based.
Therefore, we think that a file-based approach will shorten the distance between the classroom and the professional activity, which is one of our main goals.
Moreover, image-based enviromnents tend to blur the distinction between the programming environment and the program under development,
which can embody great possibilities for advanced programmers but often does not more than confuse beginners.
Also, being file-based, Wollok allows for integration for the most popular code
repositories. \jf{As well as other potential usages still not proven like
code-reviewing tools ?}

Finally, the programming enviroment incorporates several advanced characteristics from professional environments, 
which improves the coding experience, and at the same time allows the student to familiarize himself with the kind of tools he will face in his later professional activity.
For that matter, the environment incorporates content assist, automatic refactorings, advanced code navigation, 
language semantics-aware search tools and error highlighting \emph{while-you-type}

\medskip
% Contribution
It is important to notice that neither the Wollok language nor the programming environment contain novel features that are unseen in industrial tools.
Therefore, the distinctive characteristic of these tools is the search for a programming environment which 
(a) supports our pedagogical approach,
(b) feeds the student with a set of tools which are adequate to his current knowledge
and (c) gently prepares him to be using industrial-level tools.

A big amount of effort in our research has been put in looking for solutions that can solve the apparent controversy between the objectives (b) and (c).
Often, the rich set of tools an industrial language or programming environment offers, cannot be exploited by an unexperienced programmer or even confuses him.
On the other end, we think that poor programming environments fail to help the students to make his first steps in programming and so trims the possibilities of introductory courses.
Therefore, there is much to gain from a language that has the exact features a teacher wants to teacher
and a programming environment which provides the exact tools a student can take advantage off at each time of his learning process.

% Constraints that influenced the solution (because the solution is not
% universal) \emph{e.g.} our requirements for a solution, possibly not all
% satisfied. They should be sound and believable. Analysis of the criteria.
% Imagine that you are another guy having this problem do the constraint
% matches yours so that you could apply the solution
\np{Esto tenía ganas de decir pero no me doy cuenta dónde ponerlo... no sé si volarlo y ya.}
So far we have focused on university students which have had a previous subject on inperative programming.
It is a pending job to adapt these ideas to teenagers or more generally students without any prior programming experience.
