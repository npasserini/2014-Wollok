\section{Related Works}
\label{sec:related}

% Other solutions in the domain, and a real comparison of our contribution with solutions from other people.

% Agregar referencia al paper de Fidel, y otros

% Base tomada del paper de ESUG 2011
LOOP is presented as a visual environment to teach OOP using
a reduced set of language constructions and a prototype approach
to create objects. It presents the main concepts of object, message
and reference in a specialized tool with a visual representation of
the object environment. Several visual tools to teach programming
already exists, like ObjectKarel[2], Scratch[14] and Etoys[4].
ObjectKarel presents a visual tool based on the abstraction of
robots to teach OOP, using a map where the robots-the objects-
move when messages are sent to them. LOOP does not center on a
specific abstraction like a robot: it allows the student to create any
other abstraction. Scratch and Etoys, are aimed to teach the basics
of programming to children, using visual objects and scripts to play
tween objects. This kind of diagrams could be inferred from the
evaluation of any piece of code, even the execution of tests.
Another subject of research is a “debugger” for the tool [1]. We
think that a live and powerful debugger a ` la Smalltalk is a rich tool
for the understanding of the whole environment behaviour. After a
message is sent, a debugger view can be used like a video player,
with play, forward and backward buttons to navigate the message
stack and see how the state changes after each message send in the
object diagram.
Finally, there are some improvements to be made to the user
interface, such as shortcuts, code completion, improved menus or
internationalization. Currently the tool is only available in spanish,
we want to make it configurable to add more languages as neces-
sary.

In response to interest in an objects-first approach, several
texts and software tools have been published/developed that
promote this strategy (such as [1, 12]). Four recent software
tools are worthy of mention as using an objects-first
approach: BlueJ [9], Java Power Tools [11], Karel J. Robot
[2], and various graphics libraries. Interestingly, all these
tools have a strong visual/graphical component; to help the
novice “see” what an object actually is – to develop good
intuitions about objects/object-oriented programming.
BlueJ [9] provides an integrated environment in
which the user generally starts with a previously defined set
of classes. The project structure is presented graphically, in
UML-like fashion. The user can create objects and invoke
methods on those objects to illustrate their behavior. Java
Power Tools (JPT) [11] provides a comprehensive,
interactive GUI, consisting of several classes with which 
the student will work. Students interact with the GUI, and
learn about the behaviors of the GUI classes through this
interaction. Karel J. Robot [2] uses a microworld with a
robot to help students learn about objects. As in Karel [10],
Robots are added to a 2-D grid. Methods may be invoked
on the robots to move and turn them, and to have the robots
handle beepers. Bruce et al. [3] and Roberts [13] use
graphics libraries in an object-first approach. Here, there is
some sort of canvas onto which objects (e.g. 2-D shapes)
are drawn. These objects may have methods invoked on
them and they react accordingly.
In the remainder of this paper, we present a new
tactic and software support for an objects-first strategy. The
software support for this new approach is a 3D animation
tool. 3D animation assists in providing stronger object
visualization and a flexible, meaningful context for helping
students to “see” object-oriented concepts. (A more detailed
comparison of the above tools with our approach is
provided in a later section.) \cite{cooper_teaching_2003}
