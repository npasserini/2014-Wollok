\section{Related Works}
\label{sec:related}

% Other solutions in the domain, and a real comparison of our contribution with solutions from other people.

% Agregar referencia al paper de Fidel, y otros

% Base tomada del paper de ESUG 2011

The first aspect to analyse is the way an introductory course has to be,
Vilmer \etal \cite{vilner_2007} presents a work exposing the advantages of the implementation of object-first introductory courses. 
Also, Moritz \etal \cite{moritz_2005} presents a way of starting the learning of a programming language using an object-first way using multimedia and intelligent tutoring.
Another interesting work in this area is the one from Sajaniemi \etal \cite{Sajaniemi_teachingprogramming:} who presents another way to introduce the main concepts.
All this authors propose the idea of using a commercial language like Java or C\#, they are not addressing the problems of using this industrial level languages as an start point.
On the other hand, Lopez \etal \cite{lopez_nombre_2012} presents a successfully way of teaching using functional-first in an introductory course. 

Another aspect to analyse is the use of an industrial programming language or a custom one. 
In this subject, \cite{lopez_nombre_2012} \etal have chosen the same approach as us, but in a functional-first solution. Both languages centre in the main concepts of the paradigms. 
Another custom language centring in the main concepts is BlueJ \cite{bennedsen_bluej_2010}. This implementation shares with Wollok the idea to simplify the language, but it is only class centred. Wollok is both class and object centred, so there is no need to teach classes to start learning basic concepts of the paradigm.

There are interesting works in the Visual languages as a way of teaching OOP: Scratch \cite{malan_scratch_2007}, Etoys \cite{lee_empowering_2011} and Kodu \cite{kodu}. But all of them are far away of a professional development environment, so the transition to a industrial level work is not so easy as with Wollok.
