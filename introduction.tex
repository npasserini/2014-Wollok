\section{Introduction}
\label{sec:intro}

% Context
% (a) hay que arrancar introduciendo nuestra visión, contando por qué enseñamos objetos de determinada manera, por qué hicimos Ozono.

% La importancia de arrancar con objetos. -- No sé si arrancar con esto.
\emph{Object-oriented programming} (OOP) has become the \textit{de facto} standard programming paradigm in industrial software development.
Therefore, in the last years software engineering curricula have put more emphasis in object-oriented courses.
% Furthermore, opinion in the community is divided as if it should be the first programming model to be presented. 
% Problem
Still, students often have difficulties in learning how to program in an object-oriented style which show up both in academic and in industrial environments.

Several causes have been blamed for the difficulties in OOP learning.
First, OO courses tend to focus too much on syntax and the particular characteristics of a language, instead of focusing on OOP distinctive characteristics.
Second, many OO languages used in introductory courses to require to grasp a lot of quite abstract concepts before being able to build a first program.
Finaly, poor programming environments are used, in a moment when a unexperienced programmer could make great use of the guidance a good programming environment can provide.
\np{En realidad algunos de estos problemas no son exclusivos de OOP, habría que ver si queremos decir algo al respecto.}

\medskip 

% Known tracks for solutions
% here you want to show that you are not an idiot not knowing what have been around

% Primero hablar de lenguajes específicos para enseñar
There have been several proposals to address the difficulties in introductory OO courses 
by defining a specific language which provides a simplified programming model such as Karel++ \cite{bergin_karel++:_1996} \np{¿otros?}.
This approach has been used even outside the OO world \cite{feurzeig_programming-languages_1970, pattis_karel_1981, gobstones}.
% Environments
A step further is to provide a whole programming environment specifically designed to aid novice programmers 
such as Squeak \cite{ingalls_back_1997}, Traffic \cite{broy_outside-method_2003} and BlueJ \cite{bennedsen_bluej_2010}. 

% Object first
The great differences between these programming languages and environments show that they have to be analyzed in the light of the pedagogical approaches behind them.
The tools are of little use without this pedagogical view.
For example some educational languages and environments are designed to be used in \textit{object-first} approaches, 
\ie for students without any previous programming knowledge \cite{1, 3, 12, 13 del paper de cooper}.
% Children
Other languages are focused on teaching to children or teenagers such as Scratch \cite{malan_scratch_2007} and Etoys \cite{lee_empowering_2011}. 
% Visualization
and there are many approaches which emphasize the importance of \emph{visualization tools} 
to simplify the understanding of the underlying programming model\cite{cooper_teaching_2003, Java Power Tools [11]}.


\medskip
% Nuestro trabajo

Previous works of our team \cite{lombardi_instances_2007,lombardi_carlos_alumnos_2008,griggio_programming_2011,spigariol_lucas_ensenando_2013} have described an approach consisting of
(a) a novel path to introduce OO concepts, focusing on objects, messages and polymorphism, while delaying the introduction of classes and inheritance and 
(b) a reduced and graphical programming environment which supports the order in which we introduce the concepts, by allowing to build OO programs without the need of classes.
Our approach focuses on the concepts of object, message, reference and object polymorphism, while delaying the introduction of more abstract concepts such as types, classes and inheritance.
These way of organizing a course provides a more gentle learning curve to students and allows them to write programs from the first week.

% (b) lo que aprendimos en estos 8 años haciendo eso, que nos lleva a querer darle una vuelta más.
This work builds on the experience of eight years using our approach in 4 different universities,
trying to correct some faults we have identified while keeping the distinctive characteristics that made it successful.

\medskip 

% What our solution is \ct{Set} and \ct{OrderedCollection} (so that the reader knows where the paper is going)



Si bien todo esto que se nos ocurrió en su momento fue maravilloso y genial y los pibes aprenden más, programan mejor y la ponen más seguido, fuimos aprendiendo más cosas y estamos trabajando ne unos cambios:

Los fáciles:
- integrar clases y objetos en un mismo programa
- integrar interfaces de usuario automáticas
- utilizar herramientas avanzadas para guiar a los alumnos en el proceso de aprendizaje, la vedette acá sería el sistema de tipos.
- acercar la práctica de lo que hacemos a la práctica industrial (acá hay que ver qué decimos y qué no, por ejemplo me molesta que no sea un archivo... pero desde lo metodológico se puede pensar en tests o incluso en un repositorio de código... sobre la relación con la industria se podría escribir un libro, hay que ver cuánto queremos meternos).

No sé si hablar de los temas de colecciones, son un poco particulares de Ozono.


\medskip 

% Contribution of the paper
(c) las ideas nuevas... y wollok como herramienta para dar soporte a esas ideas.

Lo concreto que hicimos es tener un lenguaje con
- clases y objetos integrados
- un IDE profesional con syntax highlighting, refactors, autocompletion y la vedette (?) inferencia de tipos (en progreso).
- bueno, muchas mejoras a nivel lenguaje, como literales para coleciones, imports
- una forma fácil de construir tests.

\medskip 

% Paper structure


