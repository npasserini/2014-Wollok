\section{Discussion}
\label{sec:discussion}

% Discussion of actual solution \emph{vs.} initial constraints from \ref{sec:problem}. Explain the space of the solution, why we made it this way.
There are number of discussions that arised while evaluating the experiences
with our previous tools in courses, and that were taked into account in Wollok's
design.

One of them is whether the learning experience should provide a way to reuse
code while still in the form of pure objects without classes.
We detected that while working only with objects, on the first part of the
course, it's common to end-up with excersices where there will be at least some
sort of duplicated code between objects.
On this matter Ozono provided a mechanism based on prototypes, similar to that
found in Self language\cite{Ungar87self:the, Ungar91organizingprograms}. That
allowed full reuse of code (both behavior and state).
On the other hand, contrarily to what we suspected while designing Ozono, we've
found that in practice, introducing prototyping didn't smooth the transition
between objects and classes. Actually it made it harder.
In addition, prototyping is an excellent abstract model for pure object
languages, keeping it simple and incredible powerful. But there are almost no popular or industrial
languages implementing it (besides Javascript which is not exactly as standard
Self prototyping). 


So Wollok doesn't provide any form of inter-object reusage (besides classes).
When duplicated code appears as part of the course, it's important to
highlight that to students, and delay solving code duplication later with
classes.

\medskip


Se podría hablar de la discusión sobre si proveer formas de compartir comportamiento sin clases, como clonado.
También de la necesidad de crear colecciones independientemente, aunque es un poco específico de Ozono.

\medskip
Acercar la experiencia de aprendizaje a las prácticas industriales: 
(acá el palo de que la imagen sólo existe en smalltalk, y en la industria nadie la usa. 
Atrás de eso, la idea de archivos, y poder compartir con SVC. Por último la idea de actualizarse a un lenguaje con influencia de lenguajes modernos como xtend, scala, ruby, etc.)


% Evaluation of the solution. How does the solution meet the criteria? Where does it succeed or fails...
