\documentclass[preprint,10pt]{sigplanconf}
% \documentclass{article}
\usepackage[T1]{fontenc} %%%key to get copy and paste for the code!
\usepackage[utf8]{inputenc} %%% to support copy and paste with accents for frnehc stuff
\usepackage{times}
\usepackage[scaled=0.85]{helvet}
\usepackage{graphicx}
\usepackage{ifthen}
\usepackage{xspace}
\usepackage{alltt}
\usepackage{latexsym}
\usepackage{url}            
\usepackage{amssymb}
\usepackage{amsfonts}
\usepackage{amsmath}
\usepackage{stmaryrd}
\usepackage{enumerate}

\usepackage[pdftex,colorlinks=true,pdfstartview=FitV,linkcolor=blue,citecolor=blue,urlcolor=blue]{hyperref}
\usepackage{xspace}

\newboolean{showcomments}
\setboolean{showcomments}{true}
\ifthenelse{\boolean{showcomments}}
  {\newcommand{\bnote}[2]{
	\fbox{\bfseries\sffamily\scriptsize#1}
    {\sf\small$\blacktriangleright$\textit{#2}$\blacktriangleleft$}
    % \marginpar{\fbox{\bfseries\sffamily#1}}
   }
   \newcommand{\cvsversion}{\emph{\scriptsize$-$Id: macros.tex,v 1.1.1.1 2007/02/28 13:43:36 bergel Exp $-$}}
  }
  {\newcommand{\bnote}[2]{}
   \newcommand{\cvsversion}{}
  } 


\newcommand{\here}{\bnote{***}{CONTINUE HERE}}
\newcommand{\nb}[1]{\bnote{NB}{#1}}

\newcommand{\fix}[1]{\bnote{FIX}{#1}}
%%%% add your own macros 

\newcommand{\sd}[1]{\bnote{Stef}{#1}}
\newcommand{\np}[1]{\bnote{Nico}{#1}}
\newcommand{\gd}[1]{\bnote{Gise}{#1}}

\graphicspath{{figures/}}
%%% 


\newcommand{\figref}[1]{Figure~\ref{fig:#1}}
\newcommand{\figlabel}[1]{\label{fig:#1}}
\newcommand{\tabref}[1]{Table~\ref{tab:#1}}
\newcommand{\layout}[1]{#1}
\newcommand{\commented}[1]{}
\newcommand{\secref}[1]{Section \ref{sec:#1}}
\newcommand{\seclabel}[1]{\label{sec:#1}}

%\newcommand{\ct}[1]{\textsf{#1}}
\newcommand{\stCode}[1]{\textsf{#1}}
\newcommand{\stMethod}[1]{\textsf{#1}}
\newcommand{\sep}{\texttt{>>}\xspace}
\newcommand{\stAssoc}{\texttt{->}\xspace}

\newcommand{\stBar}{$\mid$}
\newcommand{\stSelector}{$\gg$}
\newcommand{\ret}{\^{}}
\newcommand{\msup}{$>$}
%\newcommand{\ret}{$\uparrow$\xspace}

\newcommand{\myparagraph}[1]{\noindent\textbf{#1.}}
\newcommand{\eg}{\emph{e.g.,}\xspace}
\newcommand{\ie}{\emph{i.e.,}\xspace}
\newcommand{\etal}{\emph{et al.,}\xspace}
\newcommand{\ct}[1]{{\textsf{#1}}\xspace}
\newcommand{\cf}{\emph{cf.}\xspace}

\newenvironment{code}
    {\begin{alltt}\sffamily}
    {\end{alltt}\normalsize}

\newcommand{\defaultScale}{0.55}
\newcommand{\pic}[3]{
   \begin{figure}[h]
   \begin{center}
   \includegraphics[scale=\defaultScale]{#1}
   \caption{#2}
   \label{#3}
   \end{center}
   \end{figure}
}

\newcommand{\twocolumnpic}[3]{
   \begin{figure*}[!ht]
   \begin{center}
   \includegraphics[scale=\defaultScale]{#1}
   \caption{#2}
   \label{#3}
   \end{center}
   \end{figure*}}

\newcommand{\infe}{$<$}
\newcommand{\supe}{$\rightarrow$\xspace}
\newcommand{\di}{$\gg$\xspace}
\newcommand{\adhoc}{\textit{ad-hoc}\xspace}

\usepackage{url}            
\makeatletter
\def\url@leostyle{%
  \@ifundefined{selectfont}{\def\UrlFont{\sf}}{\def\UrlFont{\small\sffamily}}}
\makeatother
% Now actually use the newly defined style.
\urlstyle{leo}


\lstset{ %
  backgroundcolor=\color{white},   % choose the background color; you must add \usepackage{color} or \usepackage{xcolor}
  basicstyle=\footnotesize\ttfamily,        % the size of the fonts that are used for the code
  breakatwhitespace=false,         % sets if automatic breaks should only happen at whitespace
  breaklines=true,                 % sets automatic line breaking
  captionpos=b,                    % sets the caption-position to bottom
  escapeinside={\%*}{*)},          % if you want to add LaTeX within your code
  extendedchars=true,              % lets you use non-ASCII characters; for 8-bits encodings only, does not work with UTF-8
  keepspaces=true,                 % keeps spaces in text, useful for keeping indentation of code (possibly needs columns=flexible)
  numbersep=5pt,                   % how far the line-numbers are from the code
  rulecolor=\color{black},         % if not set, the frame-color may be changed on line-breaks within not-black text (e.g. comments (green here))
  showspaces=false,                % show spaces everywhere adding particular underscores; it overrides 'showstringspaces'
  showstringspaces=false,          % underline spaces within strings only
  showtabs=false,                  % show tabs within strings adding particular underscores
  stepnumber=2,                    % the step between two line-numbers. If it's 1, each line will be numbered
  tabsize=2,                       % sets default tabsize to 2 spaces
}

\lstdefinelanguage{Wollok}{
  keywords={program, console},
  sensitive=true,
  comment=[l]{//},
  morecomment=[s]{/*}{*/},
  morestring=[b]',
  morestring=[b]"
}


\begin{document}
\title{Wollok -- y una frase que describa la idea}
\authorinfo{Nicolás Passerini}
  {UTN -- Facultad Regional Buenos Aires \\ Universidad Nacional de Quilmes \\ Universidad Nacional de San Martín}
  {npasserini@gmail.com}
  
\authorinfo{Javier Fernandes}
  {Universidad Nacional de Quilmes \\ Universidad Nacional de San Martín}
  {javier.fernandes@gmail.com}

\date{\today}
\maketitle

\begin{abstract}
In this context...
We consider this problem P...
P is a problem because...
We propose this solution...
Our solution solves P in such and such way.
\end{abstract}


\section{Introduction}
\label{sec:intro}

% Context
% (a) hay que arrancar introduciendo nuestra visión, contando por qué enseñamos objetos de determinada manera, por qué hicimos Ozono.

Enseñanza tradicional:
- Muchas veces se parte del lenguaje
- Bajos niveles de aprobación
- Pasa mucho tiempo hasta que un chico puede hacer un programa "real", demasiados conceptos.
- ... después pienso más

Nosotros propusimos:
- Elegir un recorrido que permite ir incorporando los conceptos uno a uno
- Tener una herramienta que da soporte a eso
- Focalizar en objeto-mensaje-polimorfismo-referencias, los demás conceptos aparecen después.


Also these hindrances reduce the opportunity of students to apply
the concepts of the paradigm effectively in their further
professional practice, resulting in several IT-projects not taking
advantage of the possibilities offered by the potential of good
object-oriented practices, even in cases where the tools used may
allow the application of object-oriented programming gracefully. \cite{lombardi_instances_2007}

Secondly, better environments have become necessary. Earlier introductory courses
focused on the development of algorithms in procedural or functional languages. To
do this, an editor and a compiler was all that was needed for the practical part of the
work. Modern courses now use object-oriented languages and subject material
taught includes testing, debugging and code reuse. This creates the need to deal with
multiple source files and multiple program development tools from the very start. To
give a beginning student a chance to cope with this increased complexity, better
environment support is needed. \cite{kolling_problem_1999}

% Problem

\medskip 

% Known tracks for solutions
We propose to provide the student a reduced and graphical
programming environment in which the object and the message are
the central concepts instead of defining classes and then instantiate
them. \cite{griggio_programming_2011}

% here you want to show that you are not an idiot not knowing what have been around


\medskip 

% What our solution is \ct{Set} and \ct{OrderedCollection} (so that the reader knows where the paper is going)

(b) lo que aprendimos en estos 8 años haciendo eso, que nos lleva a querer darle una vuelta más.

Si bien todo esto que se nos ocurrió en su momento fue maravilloso y genial y los pibes aprenden más, programan mejor y la ponen más seguido, fuimos aprendiendo más cosas y estamos trabajando ne unos cambios:

Los fáciles:
- integrar clases y objetos en un mismo programa
- integrar interfaces de usuario automáticas
- utilizar herramientas avanzadas para guiar a los alumnos en el proceso de aprendizaje, la vedette acá sería el sistema de tipos.
- acercar la práctica de lo que hacemos a la práctica industrial (acá hay que ver qué decimos y qué no, por ejemplo me molesta que no sea un archivo... pero desde lo metodológico se puede pensar en tests o incluso en un repositorio de código... sobre la relación con la industria se podría escribir un libro, hay que ver cuánto queremos meternos).

No sé si hablar de los temas de colecciones, son un poco particulares de Ozono.


\medskip 

% Contribution of the paper
(c) las ideas nuevas... y wollok como herramienta para dar soporte a esas ideas.

Lo concreto que hicimos es tener un lenguaje con
- clases y objetos integrados
- un IDE profesional con syntax highlighting, refactors, autocompletion y la vedette (?) inferencia de tipos (en progreso).
- bueno, muchas mejoras a nivel lenguaje, como literales para coleciones, imports
- una forma fácil de construir tests.

\medskip 

% Paper structure


\section{Problem Description}
\label{sec:problem}

% Context, exposed with the \textbf{most precise terms possible} (don't open
% unwanted doors for the reader)

% Probably set the vocabulary before to cut any misinterpretation

% Constraints that influenced the solution (because the solution is not
% universal) \emph{e.g.} our requirements for a solution, possibly not all
% satisfied. They should be sound and believable. Analysis of the criteria.
% Imagine that you are another guy having this problem do the constraint
% matches yours so that you could apply the solution

% Problem

This means students must dive
right into classes and objects, their encapsulation (public
and private data, etc.) and methods (the constructors,
accessors, modifiers, helpers, etc.). All this is in addition to
mastering the usual concepts of types, variables, values, and
references, as well as with the often-frustrating details of
syntax. Now, add event-driven concepts to support
interactivity with GUIs! As argued by [11], learning to
program objects-first requires students grasp "many
different concepts, ideas, and skills…almost concurrently.
Each of these skills presents a different mental challenge." \cite{cooper_teaching_2003}

% Factual solution tracks, to position...

Ozono hereda muchos de los problemas de los entornos Smalltalk

Smalltalk, however, lacks other important facilities: no visualisation tools for class
relations are available. The main problem with this lies in the Smalltalk language
itself: since it is not statically typed, it is not possible to extract usage relations from
its source code. No indication exists before runtime as to the call relationships
between classes. Inheritance relationships as shown in the browser do not present
the relationships of one application but rather the whole Smalltalk environment and
so the browser is not used as an application modelling tool. Smalltalk blurs the
distinction between the environment and the application under development.
Reports about the use of Smalltalk systems for teaching also point to another
problem: its size. While the language itself (in terms of the number of constructs) is
small, the class library and tools are large and often confusing. Several authors
reported difficulties with the students ability to cope with the environment [7, 8],
especially that experimentation and self directed learning was not working well
because students were overwhelmed by the system. They also found that the
functionality of the browser should be limited, since its power and flexibility caused
more problems than it solved. \cite{kolling_problem_1999}.

% Our solution in a nutshell.


\section{A Customized Programming Environment}
\label{sec:environment}

Beginner programmers are likely to require more guidance and make more mistakes than experienced programmers.
Therefore, we think that is much to gain from a good programming environment which structures the programming experienced and helps the students to identify common mistakes.

% \subsubsection{Visualización y Navegación}
% \begin{itemize}
%   \item syntax highlight
%   \item outline
%   \item hovering
%   \item vista de problems 
%   \item navigate: goto (F3, click), flechita para ir al método que sobrescribe.
%   \item find references 
% \end{itemize}
% 
% \subsubsection{Asistencia}
% \begin{itemize}
%   \item content assist
%   \item quick fixes
%   \item code templates (nuevo)
% \end{itemize}
The Wollok programming environment includes a lot of features that provide guidance to the student.
\emph{Content assist} shows the students what are his possibilities at any moment and feeds automatically into the code the most usual constructs, 
allowing the student to concentrate less on syntax and more in the modelling of the exercise problem.
\emph{Quick-fixes} allow Wollok not only to highlight problems in the student's code but also to propose automatic solutions for some usual mistakes.
\emph{Advanced code navigation} and \emph{smart reference searches} allow the programmer to better understand the dependencies in his program.
\np{¿Se les ocurre cómo mejorar eso?}
Moreover, \emph{automatic class diagrams} provide a high level view of the program and also helps understanding.

% Detect mistakes
Also, the programming environment has many tools intended to help detecting mistakes, even while the student is writing code.
\emph{Syntax highlighting} helps identify the most simple mistakes by providing immediate feedback when something is not right. 
Moreover, the enviromnent provides \emph{real-time highlights} for syntactic mistakes.
Finally, the \emph{type inferer} allows to detect more subtle mistakes.
All these tools allows the student to gain more control of his code, keeping him away from feeling lost, 
which is otherwise a common situation for a student walking his first steps into programming.

% Este no sé cómo ponerlo, es muy crítica al smalltalk.
% 8-reducir errores frustrantes: se cancela la edicion por tener 1 solo editor de metodo por ves (poder visualizar más que un sólo método simul), evitar errores de imagenes)

\medskip
% Type inferer
The type inferer is one of the most distintive characteristics of the Wollok programming environment.
We think that type inference is key to a simple programming environment.
On one side, it allows to detect lots of common mistakes \emph{before running the program}:
if an object does understand a message, if a wrong argument is passed, if incompatible types are mixed or even miss-spellings.
In enviroments without this capability it takes more time to detect errors.
Moreover, it is not uncommon that a type mistake produces a runtime error in a place different from where the mistake was done, producing confussion.

Still, providing a type inferer for a language such as Wollok has many subleties, which deserves an independent study \cite{type inferer}.
On one side we require it to be able to work without type annotations and at the same time provide feedback useful for an unexperienced programmer.
On the other side, the type system is rather complex;
for example, the presence of stand-alone objects requires the type system to handle \emph{structural types}, since a named type system would not allow them to be treated polymorphically.
Also, we want to be able to treat polymorphically stand alone objects with class-based objects.

\subsubsection{Checkeos y validaciones}
 
Todos los checkeos y problemas
generados se muestran agrupados en una vista dedicada a tal fín (Problems).

% En lo que sigue fui comentando las cosas que ya están dichas pero no quiero borrar esta enumeración porque está muy buena.
\begin{itemize}
%   \item \textbf{De sintaxis}: dados por el parser y lexer automáticamente.
  \item \textbf{De estilo}: para promover uniformidad y consistencia de código.
  Ejemplos:
  		\begin{itemize}
  			\item \textit{Nombres}: variables camelcase comenzando en minúscula,
  			nombres de clases camelcase iniciando mayúscula, packages en minúsculas, etc.
  			\item \textit{Orden y agrupamiento}: dentro de un objeto o clase, primero
  			se definen sus referencias internas, luego constructores y finalmente los métodos.
  			\item \textit{Separación de programas}: las clases sólo se pueden definir
  			en archivos de tipo \textit{librería}, no dentro de un \textit{program}.
  			\item \textit{Evitar referencias duplicadas}: no se puede definir una
  			referencia con nombre ya utilizado en alguna otra referencia del contexto (local, método,
  			clase/objeto, etc.). Ni tampoco si ya está definida en la superclase.
		\end{itemize}
  \item \textbf{De resolución de referencias}: para evitar referencias a
  variables inexistentes y, en la medida de lo posible (por ser de tipado
  implícito) de envío de mensajes. Ejemplos:
  		\begin{itemize}
		  \item \textit{Referencias inexistentes}: a variables locales, parámetros, o
		  internas (clase/objeto).
		  \item \textit{Constructores inexistentes}: evaluando existencia de la
		  clase, y compatibilidad en el número de paråmetros.
		  \item \textit{Envío de mensajes (a this)}: al ser a this se pueden realizar
		  checkeos por la existencia del método y compatibilidad de parámetros, incluso sin
		  involucrar al sistema de tipos.
		\end{itemize}
  \item \textbf{De uso de referencias}: para la detección de código
  	erroneo o bien desactualizado. Por ejemplo: warnings por referencias nunca
 	utilizadas, nunca asignadas, o utilización de variables en lugar de valores.
  \item \textbf{De estructura}: evitan por ejemplo inconsistencias en las
  estructuras creadas por el alumno. Por ejemplo, se checkea
  que un método marcado como \textit{override} efectivamente esté
	sobrescribiendo.
  \item \textbf{De tipos}: verifican compatibilidad de referencias en base a sus
  tipos. Por ejemplo ante envío de mensajes, o asignaciones de variables. Basado
  en el sistema de tipos.
\end{itemize}

\np{override}


\section{Discussion}
\label{sec:discussion}

% Discussion of actual solution \emph{vs.} initial constraints from \ref{sec:problem}. Explain the space of the solution, why we made it this way.

% Evaluation of the solution. How does the solution meet the criteria? Where does it succeed or fails...

\section{Related Works}
\label{sec:related}

% Other solutions in the domain, and a real comparison of our contribution with solutions from other people.

% Agregar referencia al paper de Fidel, y otros

% Base tomada del paper de ESUG 2011
LOOP is presented as a visual environment to teach OOP using
a reduced set of language constructions and a prototype approach
to create objects. It presents the main concepts of object, message
and reference in a specialized tool with a visual representation of
the object environment. Several visual tools to teach programming
already exists, like ObjectKarel[2], Scratch[14] and Etoys[4].
ObjectKarel presents a visual tool based on the abstraction of
robots to teach OOP, using a map where the robots-the objects-
move when messages are sent to them. LOOP does not center on a
specific abstraction like a robot: it allows the student to create any
other abstraction. Scratch and Etoys, are aimed to teach the basics
of programming to children, using visual objects and scripts to play
tween objects. This kind of diagrams could be inferred from the
evaluation of any piece of code, even the execution of tests.
Another subject of research is a “debugger” for the tool [1]. We
think that a live and powerful debugger a ` la Smalltalk is a rich tool
for the understanding of the whole environment behaviour. After a
message is sent, a debugger view can be used like a video player,
with play, forward and backward buttons to navigate the message
stack and see how the state changes after each message send in the
object diagram.
Finally, there are some improvements to be made to the user
interface, such as shortcuts, code completion, improved menus or
internationalization. Currently the tool is only available in spanish,
we want to make it configurable to add more languages as neces-
sary.

In response to interest in an objects-first approach, several
texts and software tools have been published/developed that
promote this strategy (such as [1, 12]). Four recent software
tools are worthy of mention as using an objects-first
approach: BlueJ [9], Java Power Tools [11], Karel J. Robot
[2], and various graphics libraries. Interestingly, all these
tools have a strong visual/graphical component; to help the
novice “see” what an object actually is – to develop good
intuitions about objects/object-oriented programming.
BlueJ [9] provides an integrated environment in
which the user generally starts with a previously defined set
of classes. The project structure is presented graphically, in
UML-like fashion. The user can create objects and invoke
methods on those objects to illustrate their behavior. Java
Power Tools (JPT) [11] provides a comprehensive,
interactive GUI, consisting of several classes with which 
the student will work. Students interact with the GUI, and
learn about the behaviors of the GUI classes through this
interaction. Karel J. Robot [2] uses a microworld with a
robot to help students learn about objects. As in Karel [10],
Robots are added to a 2-D grid. Methods may be invoked
on the robots to move and turn them, and to have the robots
handle beepers. Bruce et al. [3] and Roberts [13] use
graphics libraries in an object-first approach. Here, there is
some sort of canvas onto which objects (e.g. 2-D shapes)
are drawn. These objects may have methods invoked on
them and they react accordingly.
In the remainder of this paper, we present a new
tactic and software support for an objects-first strategy. The
software support for this new approach is a 3D animation
tool. 3D animation assists in providing stronger object
visualization and a flexible, meaningful context for helping
students to “see” object-oriented concepts. (A more detailed
comparison of the above tools with our approach is
provided in a later section.) \cite{cooper_teaching_2003}


\section{Conclusion}
\label{sec:conclusion}

% In this paper, we looked at problem P with this context and these
% constraints. We proposed solution S. It has such good points and such not so
% good ones. 

% Now we could do this or that.

Y a futuro le agregaríamos:
- Integración con la UI
- Integración fácil con SCM

Another characteristic of programming in the real world is the need to work in
teams. The success of object-oriented languages is partly due to their advantages in
group projects. Ideally, we also want to teach our students about the techniques
needed for teamwork. To do this, it is essential that the environment has some form
of support for group work. \cite{kolling_problem_1999}

- Más refactors y mejora del sistema de inferencia.
- Una versión web / versión liviana con el objetivo de poder ejecutarse en las netbooks que tienen los chicos de secundaria.

Probar la nueva herramienta en entornos educativos.

Eso pensando el lenguaje/herramienta, si pienso a nivel docencia se me ocurre que lo que tenemos que hacer es integrarnos con otras entidades, como Sadosky u otras universidades.
Haciendo foco en que el punto no es la herramienta sino que tenemos que repensar cómo enseñamos a programar.

\subsection*{Acknowledgements} 
% This work was supported by Ministry of Higher Education and Research, Nord-Pas de Calais Regional Council, FEDER through the 'Contrat de
% Projets Etat Region (CPER) 2007-2013',  the Cutter ANR project, ANR-10-BLAN-0219 and the MEALS Marie Curie Actions program FP7-PEOPLE-2011-
% IRSES MEALS (no. 295261). 

% \bibliographystyle{plain}
% \bibliography{foo.bib}

% \appendix
% 
% \section{Lots of Furry Technical Details}

{
\small
\bibliographystyle{abbrv}
\bibliography{wollok,teaching,scg}
}

\newpage
\appendix
\section{Images}
Imágenes y otros detalles de wollok que no entran en las 6/7 páginas del artículo
\end{document}
